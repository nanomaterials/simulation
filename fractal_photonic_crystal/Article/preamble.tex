\usepackage{etex}
\reserveinserts{28}

\usepackage{comment}
\usepackage{lipsum} % for some dummy text

%%% Работа с русским языком
\usepackage{cmap}					% поиск в PDF
\usepackage{mathtext} 				% русские буквы в фомулах
\usepackage[T1,T2A]{fontenc}
\usepackage[utf8]{inputenc}			% кодировка исходного текста
\usepackage[english,russian]{babel}	% локализация и переносы

%%% Дополнительная работа с математикой
\usepackage{amsmath,amsfonts,amssymb,amsthm}
\usepackage{mathtools}
\usepackage{icomma} % "Умная" запятая: $0,2$ --- число, $0, 2$ --- перечисление

%% Номера формул
\mathtoolsset{showonlyrefs=true} % Показывать номера только у тех формул, на которые есть \eqref{} в тексте.

%% Шрифты
\usepackage{euscript}	 % Шрифт Евклид
\usepackage{mathrsfs}    % Красивый матшрифт
\usepackage{gensymb}     % Основные символы (degree sign)
\usepackage{csquotes}

% Таблицы
\usepackage{tabularx}
\usepackage{tabulary}

% Гиперссылки
\usepackage{hyperref}
\usepackage[usenames,dvipsnames,svgnames,table,rgb]{xcolor}
\hypersetup{				% Гиперссылки
	unicode=true,           % русские буквы в раздела PDF
	pdftitle={Конспект по курсу Наноматериалы},   % Заголовок
	pdfauthor={Л.Б.Матюшкин, \\ асс. каф. микро- и наноэлектроники СПбГЭТУ <<ЛЭТИ>>},      % Автор
	pdfsubject={Тема},      % Тема
	pdfcreator={Создатель}, % Создатель
	pdfproducer={Производитель}, % Производитель
	pdfkeywords={keyword1} {key2} {key3}, % Ключевые слова
	colorlinks=true,       	% false: ссылки в рамках; true: цветные ссылки
	linkcolor=blue,          % внутренние ссылки
	citecolor=blue,         % на библиографию
	filecolor=magenta,      % на файлы
	urlcolor=blue           % на URL
}

%%% Список условных обозначений
\usepackage{glossaries}
\makeglossary    % Закомментируйте, если перечень не нужен

%%% Изображения
\usepackage{graphicx, color}
\graphicspath{{../images/}}

%%%Работа с цветом текста
\usepackage{color}
\newcommand{\hl}[1]{\colorbox{yellow}{#1}} % подсветка желтым


% Список условных обозначений
\usepackage{nomencl}
\makenomenclature
\newcommand*{\nom}[2]{#1\nomenclature{#1}{#2}}
\makeatletter
\patchcmd{\thenomenclature}{\section*}{\section}{}{}
\makeatother
\renewcommand{\nomname}{Перечень условных обозначений}

%%% Предметный указатель
\usepackage{makeidx}  % Составление предметного указателя
\makeindex
%\usepackage{showidx} % Показать части предметного указателя на правом поле

\frenchspacing

% Библиотека для добавления текста к рисунку
\usepackage[percent]{overpic}
\usepackage{nicefrac}

\clubpenalty = 10000

%%% Простая графика
\usepackage{tikz}
\usetikzlibrary{arrows, mindmap, trees, shadows, shapes, calc, fadings, positioning, decorations.pathreplacing, intersections}
\tikzset{>=latex}

%% Свои команды
\DeclareMathOperator{\sgn}{\mathop{sgn}}

\newcommand*{\hm}[1]{#1\nobreak\discretionary{}
	{\hbox{$\mathsurround=0pt #1$}}{}}

%% Оформление химических формул
\newcommand*\chem[1]{\ensuremath{\mathrm{#1}}}

\renewcommand{\ge}{\geqslant}
\renewcommand{\le}{\leqslant}